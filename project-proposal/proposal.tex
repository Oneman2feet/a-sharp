\documentclass{article}
\usepackage[top=1.2in, bottom=1in, left=2in, right=2in]{geometry}
\usepackage{enumerate, multicol}
\usepackage{amsmath}
\usepackage{parskip}

\newcommand{\HRule}{\rule{\linewidth}{0.5mm}}

\begin{document}

\begin{titlepage}
\begin{center}

\textsc{\LARGE Cornell University}\\[1.5cm]

\textsc{\Large CS 4621 Practicum Project Proposal}\\[0.5cm]

% Title
\HRule \\[0.4cm]
{ \huge \bfseries A$\sharp$ -- Music Visualizer \\[0.4cm] }

\HRule \\[1.5cm]

% Group members
\begin{minipage}{0.4\textwidth}
\begin{flushleft} \large
Shane \textsc{Moore} \\
\emph{swm85}
\end{flushleft}
\end{minipage}
\begin{minipage}{0.4\textwidth}
\begin{flushright} \large
Zachary \textsc{Zimmerman} \\
\emph{ztz3}
\end{flushright}
\end{minipage}
\par\vspace{0.5cm}
\begin{minipage}{0.4\textwidth}
\begin{flushleft} \large
Emre \textsc{Findik} \\
\emph{ef343}
\end{flushleft}
\end{minipage}
\begin{minipage}{0.4\textwidth}
\begin{flushright} \large
Joseph \textsc{Vinegrad} \\
\emph{jav86}
\end{flushright}
\end{minipage}

\vfill

% Bottom of the page
{\large October 9, 2014}

\end{center}

\end{titlepage}

%%%%%%%%%%%%%%%%%%%%%%%%%%%%%%%%%%%%%%%%%%

\section{Summary}

Our group proposes to create a music visualizer called A$\sharp$ (A Sharp) which goes beyond the interpretive scope of current visualization software.  We find that the typical music visualization does fine at looking good alongside the music, but fails to go beyond and provide actual interpretation or insight into the song.

As Edward Tufte said in his book \textit{The Visual Display of Quantitative Information}, ``At their best, graphics are instruments for reasoning about quantitative information'' (Introduction).  The job of a music visualizer, therefore, is to provide the user with enough relevant and useful visual information that they may interpret, on a higher level, the characteristics of the sound which is being visualized.  If possible, the visualization could be considered a summary of the song, or even a rudimentary alternative.  We plan to work towards this standard in our music visualizer, A$\sharp$.

\section{Software description}

The software we are planning to create will play an animation, given an audio file input. The visuals in the animation will be built upon not only instantaneous properties of the audio but will also be dependent upon qualities such as tempo, rhythm progressions and the musical genre, which requires analyses of the audio as a whole or in greater parts. Each of these audio qualities will be linked to a specific visual quality in the animation, such as the color properties (hue, saturation etc.) and the output shape.

The visualization itself will be a single sphere in the center of the screen, whose properties will change based on temporal changes in various components of the music. Most notably, displacement shaders will be used to adjust the surface texture of the sphere, allowing it to have a drastically different shape if necessary (e.g. long, varied spikes). Through manipulation of the shaders over time, using splines, we will be able to create an animation that accurately represents the music being listened to.

\section{Application in Graphics}

The graphics portion of our project will focus primarily on modeling and animation.
 
Given our goal of creating complex visual displays that accurately reflect different types of music, modeling will be a key component of our work. Determining what combination of objects, textures and shading models to use for different songs is an interesting challenge. Our plan is to begin with a single perfect sphere that we will transform as the song progresses. In addition to modeling the sphere, we will also keep track of some underlying mesh representing the transformed sphere. We will likely represent this mesh using an indexed triangle set, potentially with optimizations such as triangle strips and triangle fans. Since we are only using one object in our visualizer, we do not anticipate rendering to be a major challenge, giving us an opportunity to focus more on modeling. 
 
Animation is another major area of focus for our project, as we aim to produce a dynamic moving display for our visualizer. To enhance the design and versatility of our visualizer, we will utilize different animation patterns corresponding to different types of music. Currently our basic idea is to perform transformations on the surface of a sphere with different speeds, shapes and magnitudes depending on the music. To implement this, we will apply techniques such as bump mapping along with splines to seamlessly animate the sphere frame by frame. While we haven’t yet discussed animation in class, we plan to do our own research on animation topics before they are introduced in class.

\section{Properties of Music for Quantification}

\begin{itemize}
	\item Song genre
	\item Amplitude at a given frequency
	\item Sound location (stereo)
	\item Sound quality (Hilbert scope)
	\item Tremolo
	\item Centroid, spread, skewness and kurtosis (of an amplitude envelope)
	\item Mel-frequency cepstrum
	\item Rhythm complexity
	\item Distinguish accompaniment from melody
	\item Melody: pitch, volume, position in chord
\end{itemize}

\section{Software Architecture}

Our code will be broken up into four modules, which each represent a specific part of the program. They are laid out as follows: \\

\begin{enumerate}
	\item Data collection / statistical analysis
	\begin{itemize}
		\item Takes in sound file
		\item Runs library calls for music analysis
		\item Returns the results of all analyses in the formats of the respective libraries
	\end{itemize}
	\item Data formatting (models)
	\begin{itemize}
		\item Takes in the raw analysis data
		\item Create appropriate data structure for graphical representation
		\begin{itemize}
			\item Includes generating shaders and splines based input data 
		\end{itemize}
	\end{itemize}
	\item Controller
	\begin{itemize}
		\item Acts as a layer of abstraction between the model and view
		\item Runs the rendering engine
	\end{itemize}
	\item View
	\begin{itemize}
		\item The rendering/animation engine
		\item Given the shaders and splines, plays the animation of the sphere
	\end{itemize}
\end{enumerate}

The first module will perform some primary analyzation of the sound file that's been passed to the software. This will include attempts to determine the key and tempo of the piece, variations (and base-levels) of volume, and spectral analysis for some understanding of the instrumentation being used. Data from the analysis will be stored in the next module; a series of models that will be designed for fast access and calculations necessary for our representations of the piece.
	A controller will act as an interface between the models and views, which will store data about the visual representation itself. A final module, the renderer, will take these software representations of the animation (the views), and produce a video to accompany the song.

We plan on using python for coding this project, with a few third-party libraries for the sound analysis module.

\section{Work Distribution}

\paragraph{Emre:} Sound file analysis (\oldstylenums{1}) and model for representation of sound (\oldstylenums{2})
\vspace{-0.5cm}
\paragraph{Shane:} Model design (\oldstylenums{2}) and controllers (\oldstylenums{3})
\vspace{-0.5cm}
\paragraph{Zachary:} Sound data interpretation (\oldstylenums{3}) and visualization design (\oldstylenums{4})
\vspace{-0.5cm}
\paragraph{Joey:} Graphical representation of the view and rendering (\oldstylenums{4})

\section{Milestone}

By the milestone, our group hopes to have the following as part of a proof of concept:

\begin{itemize}
	\item Warmup: midi support, later on, wav support
	\item Working interfaces for each of the sound libraries we choose to use
	\item Integration of data from all of the libraries into one format
	\item A rudimentary visual representing every meaningful piece of insight we can glean from the music
\end{itemize}

\end{document}
