\documentclass{article}
\usepackage[top=1.2in, bottom=1in, left=2in, right=2in]{geometry}
\usepackage{enumerate, multicol}
\usepackage{amsmath}
\usepackage{parskip}

\newcommand{\HRule}{\rule{\linewidth}{0.5mm}}

\begin{document}

\begin{titlepage}
\begin{center}

\textsc{\LARGE Cornell University}\\[1.5cm]

\textsc{\Large CS 4621 Practicum Project Proposal}\\[0.5cm]

% Title
\HRule \\[0.4cm]
{ \huge \bfseries A$^\sharp$ -- Music Visualizer \\[0.4cm] }

\HRule \\[1.5cm]

% Group members
\begin{minipage}{0.4\textwidth}
\begin{flushleft} \large
Shane \textsc{Moore} \\
\emph{swm85}
\end{flushleft}
\end{minipage}
\begin{minipage}{0.4\textwidth}
\begin{flushright} \large
Zachary \textsc{Zimmerman} \\
\emph{ztz3}
\end{flushright}
\end{minipage}
\par\vspace{0.5cm}
\begin{minipage}{0.4\textwidth}
\begin{flushleft} \large
Emre \textsc{F�nd�k} \\
\emph{ef343}
\end{flushleft}
\end{minipage}
\begin{minipage}{0.4\textwidth}
\begin{flushright} \large
Joseph \textsc{Vinegrad} \\
\emph{jav86}
\end{flushright}
\end{minipage}

\vfill

% Bottom of the page
{\large October 9, 2014}

\end{center}

\end{titlepage}

%%%%%%%%%%%%%%%%%%%%%%%%%%%%%%%%%%%%%%%%%%

\section{Summary}

Our group proposes to create a music visualizer called A$\sharp$ (A Sharp) which goes beyond the interpretive scope of current visualization software.  We find that the typical music visualization does fine at looking good alongside the music, but fails to go beyond and provide actual interpretation or insight into the song.

As Edward Tufte said in his book \textit{The Visual Display of Quantitative Information}, ``At their best, graphics are instruments for reasoning about quantitative information'' (Introduction).  The job of a music visualizer, therefore, is to provide the user with enough visual information to interpret, on a higher level, the characteristics of the sound which is being visualized.  If possible, the visualization could be considered a summary of the song, and even a crude alternative.

We plan to work towards this standard in our music visualizer, A$\sharp$.

\section{Software description}

\begin{itemize}
	\item Application that takes a song file and outputs a video file to accompany it
	\item We should think about this, maybe we want to input a song, have it do preprocessing, and then immediately play the video?
	\item Perhaps also take midi input as a warmup (this is a good idea)
\end{itemize}

\section{Application in Graphics}

Two areas that we�ll cover and graphics techniques that we�ll use (video rendering)

\section{Software Architecture}

Code in python
Modules
Sound file analyzation
Model for representing different parts of sound
Timbre, key/�emotion�, amplitude
Controller acts as abstraction layer between model and view
View (graphical representation)
Software representation of the visualizer
Renderer
Actually render (and play?) the video of the visualization

Also breakdown of work

\section{Milestone}

What we�ll have completed for the milestone

Proof of concept - getting data from sound file and outputting it in some manner
Basic Architecture

\end{document}