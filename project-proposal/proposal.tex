\documentclass{article}
\usepackage[top=1.2in, bottom=1in, left=2in, right=2in]{geometry}
\usepackage{enumerate, multicol}
\usepackage{amsmath}
\usepackage{parskip}

\newcommand{\HRule}{\rule{\linewidth}{0.5mm}}

\begin{document}

\begin{titlepage}
\begin{center}

\textsc{\LARGE Cornell University}\\[1.5cm]

\textsc{\Large CS 4621 Practicum Project Proposal}\\[0.5cm]

% Title
\HRule \\[0.4cm]
{ \huge \bfseries A$^\sharp$ -- Music Visualizer \\[0.4cm] }

\HRule \\[1.5cm]

% Group members
\begin{minipage}{0.4\textwidth}
\begin{flushleft} \large
Shane \textsc{Moore} \\
\emph{swm85}
\end{flushleft}
\end{minipage}
\begin{minipage}{0.4\textwidth}
\begin{flushright} \large
Zachary \textsc{Zimmerman} \\
\emph{ztz3}
\end{flushright}
\end{minipage}
\par\vspace{0.5cm}
\begin{minipage}{0.4\textwidth}
\begin{flushleft} \large
Emre \textsc{Findik} \\
\emph{ef343}
\end{flushleft}
\end{minipage}
\begin{minipage}{0.4\textwidth}
\begin{flushright} \large
Joseph \textsc{Vinegrad} \\
\emph{jav86}
\end{flushright}
\end{minipage}

\vfill

% Bottom of the page
{\large October 9, 2014}

\end{center}

\end{titlepage}

%%%%%%%%%%%%%%%%%%%%%%%%%%%%%%%%%%%%%%%%%%

\section{Summary}

Our group proposes to create a music visualizer called A$\sharp$ (A Sharp) which goes beyond the interpretive scope of current visualization software.  We find that the typical music visualization does fine at looking good alongside the music, but fails to go beyond and provide actual interpretation or insight into the song.

As Edward Tufte said in his book \textit{The Visual Display of Quantitative Information}, ``At their best, graphics are instruments for reasoning about quantitative information'' (Introduction).  The job of a music visualizer, therefore, is to provide the user with enough relevant and useful and visual information that they may interpret, on a higher level, the characteristics of the sound which is being visualized.  If possible, the visualization could be considered a summary of the song, and even a rudimentary alternative.

We plan to work towards this standard in our music visualizer, A$\sharp$.

\section{Software description}

\begin{itemize}
	\item Application that takes a song file and outputs a video file to accompany it
	\item We should think about this, maybe we want to input a song, have it do preprocessing, and then immediately play the video?
	\item Perhaps also take midi input as a warmup (this is a good idea)
\end{itemize}

\section{Application in Graphics}

The graphics portion of our project will focus primarily on rendering and animation.
 
Given our goal of creating complex visual displays that accurately reflect different types of music, rendering will be a key component of our work. We will likely want to incorporate a variety of shapes, textures, and lighting. In order to accomplish this, we will apply techniques such as texture mapping and shading (under various models). Additionally, our scenes will probably consist of several objects on the screen simultaneously. To account for this, we will need to use ray tracing techniques to determine relative positions of objects. Still, determining what combination of objects and shading models to use for different songs remains a challenge.
 
Animation is another major area of focus for our project, as we aim to produce a dynamic moving display for our visualizer. To enhance the design and versatility of our visualizer, we will utilize different animation patterns corresponding to different types of music. To implement these, we will apply techniques related to both linear and nonlinear transformations (in 2D and 3D), as well as optimizations for seamlessly animating frame by frame. While we haven�t yet discussed animation in class, we are scheduled to do so shortly before our first milestone, giving us time to incorporate animation techniques thereafter.

\section{Software Architecture}

Code in python
Modules
Sound file analyzation
Model for representing different parts of sound
Timbre, key/``emotion'', amplitude
Controller acts as abstraction layer between model and view
View (graphical representation)
Software representation of the visualizer
Renderer
Actually render (and play?) the video of the visualization

Also breakdown of work

\subsection{Properties of Music for Quantification}

\begin{itemize}
	\item Song genre
	\item Amplitude at a given frequency
	\item Sound location (stereo)
	\item Sound quality (Hilbert scope)
	\item Tremolo
	\item Centroid, spread, skewness and kurtosis (of an amplitude envelope)
	\item Mel-frequency cepstrum
	\item Rhythm complexity
	\item Distinguish accompaniment from melody
	\item Melody: pitch, volume, position in chord
\end{itemize}

\section{Milestone}

By the milestone, our group hopes to have the following as part of a proof of concept:

\begin{itemize}
	\item Warmup: midi support, later on, wav support
	\item Working interfaces for each of the sound libraries we choose to use
	\item Integration of data from all of the libraries into one format
	\item A rudimentary visual representing every meaningful piece of insight we can glean from the music
\end{itemize}

\end{document}
